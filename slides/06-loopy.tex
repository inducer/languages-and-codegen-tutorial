\documentclass[english,compress]{beamer}
\nonstopmode%
%\batchmode
\usepackage{kloeckislides}
\nonstopmode

\usepackage{pifont}

\useoutertheme{split}
\useinnertheme{rectangles}
\usecolortheme{uiuc}
\usetikzlibrary{arrows}

\usepackage{ifthen}

\pgfdeclareimage[height=0.8cm]{uiuc-logo}{uiuc-logo.pdf}
\def\mylogotext{\pgfuseimage{uiuc-logo}\hspace*{0.3cm}}
%\def\mylogotext{}

\AtBeginSection[] {
  \begin{frame}<beamer>
  \frametitle{Outline}
  \tableofcontents[sectionstyle=show/shaded,subsectionstyle=show/show/hide]
\end{frame}
}
\AtBeginSubsection[] {
  \begin{frame}<beamer>
  \frametitle{Outline}
  \tableofcontents[sectionstyle=show/shaded,subsectionstyle=show/shaded/hide]
\end{frame}
}

\definecolor{green}{RGB}{0, 180, 0}
\definecolor{red}{RGB}{180, 0, 0}
\colorlet{grellow}{green!50!yellow}
\colorlet{codeback}{gray!20}

\DeclareMathOperator{\argmin}{argmin}
\DeclareMathOperator{\argmax}{argmax}


\lstset{
  language=Python,
  alsolanguage=C,
  rangebeginprefix=\#\ ,
  rangeendprefix=\#\ ,
  }

\colorlet{input}{green!30}
\colorlet{output}{red!30}
\colorlet{intermed}{blue!30}

\definecolor{fetch}{RGB}{227,110,35}
\definecolor{alu}{RGB}{255,188,24}
\definecolor{context}{RGB}{132,146,175}


\setbeamertemplate{navigation symbols}{}

\let\tmop=\operatorname

\usepackage[normalem]{ulem}

\def\curl{\operatorname{curl}}

\def\checkmark{\textbf{\color{green}\ding{51}}}
\def\crossmark{\textbf{\color{red}\ding{56}}}


\logoenable%

\pgfdeclarelayer{grid}
\pgfsetlayers{background,grid,main,foreground}
\def\intd{\, d}

\def\bigncentered#1{
  \begin{center}
    \Huge\bfseries #1
  \end{center}
}

\begin{document}

\title{%
   Part 6: Loopy
}

\institute{Computer Science $\cdot$ University of Illinois at
Urbana-Champaign}

\author{Andreas Klöckner}

\date{}

\frame{\titlepage}

% \begin{frame}{Thanks}
% 
%   \begin{itemize}
%   \item Tim Warburton (Rice)
%   \item Lucas Wilcox (NPS)
%   \item Leslie Greengard (NYU)
%   \item Early adopters (Rob Kirby, Maxim Kuznetsov, Ivan Oseledets)
%   %\item PyOpenCL, PyCUDA contributors
%   \item AMD, Nvidia
%   \end{itemize}
% 
% \end{frame}
% -----------------------------------------------------------------------------
\section[Loo.py]{Loop Generation}
% -----------------------------------------------------------------------------
\subsection{Loo.py}
% -----------------------------------------------------------------------------
% {{{
\begin{frame}{Automating GPU Programming}
  \begin{beamercolorbox}[sep=3mm]{block body}
    High-performance programming can be a time-consuming trial-and-error
    process.
  \end{beamercolorbox}
  Obvious idea: Let the computer do it. How?
  \begin{itemize}
    \item One way: ``Smart'' compiler, ``dumb'' developer
      \begin{itemize}
        \item GPU programming requires complex tradeoffs
        \item Tradeoffs require heuristics
        \item Heuristics are fragile
      \end{itemize}
    \item Another way: ``Smart'' developer, ``dumb'' compiler
      \begin{itemize}
        \item Error-prone
        \item Expensive in developer time
        \item User can use manual/automatic tuning
      \end{itemize}
  \end{itemize}
  \uncover<2->{%
    \begin{tikzpicture} [overlay]
      \node at (current page.center) [draw,drop shadow,fill=white,
      inner xsep=0.5cm,inner ysep=0.5cm,thick]
        {%
          So compromise!

          Following: an idea of a compromise.
        } ;
    \end{tikzpicture}
  }
\end{frame}
% -----------------------------------------------------------------------------
\begin{frame}{Setting the Stage}
  \begin{columns}
    \column{0.55\textwidth}
      Idea: Create IR + library of transformations
      \begin{itemize}
        \item Start with math-y statement of the operation
        \item ``Push a few buttons'' to optimize for the target
          device
        \item Strongly separate these two parts
      \end{itemize}

      \medskip
      Philosophy:
      \begin{itemize}
        \item Avoid ``intelligence''
        \item User can assume partial responsibility for correctness
        \item Embedding in Python provides generation/transform
          flexibility
      \end{itemize}
    \column{0.45\textwidth}
      \includegraphics[width=\textwidth]{loopy-crop.pdf}
  \end{columns}
  \uncover<2>{%
    \begin{tikzpicture} [overlay]
      \node [above left=1cm of current page.south east, draw,drop shadow,fill=white,
      inner xsep=0.5cm,inner ysep=0.5cm,thick, text
      width=0.7\textwidth]
        {%
          Loopy is infrastructure.

          \medskip
          Auto-tuners and domain-specific
          libraries are ``above'' loopy conceptually.
        } ;
    \end{tikzpicture}
  }
\end{frame}
% -----------------------------------------------------------------------------
\begin{frame}
  \bigncentered{DEMO TIME}
\end{frame}
% -----------------------------------------------------------------------------
\begin{frame}{Capturing Variants}
  \lstinputlisting[basicstyle=\scriptsize]{loopy-variants.py}
  \uncover<2>{%
    \begin{tikzpicture} [overlay]
      \node [above left=1cm of current page.south east, draw,drop shadow,fill=white,
      inner xsep=0.5cm,inner ysep=0.5cm,thick,
      text width=0.5\textwidth]
        {%
          Easy to \emph{non-redundantly} capture multiple variants of
          the same kernel.
        } ;
    \end{tikzpicture}
  }
\end{frame}
% -----------------------------------------------------------------------------
\begin{frame}{Ordering}
  \begin{itemize}
    \item Completely \emph{un}ordered by default
    \item Program only well-formed

      \emph{if} domain traversal order does not matter
    \item Depdencies

      \emph{can} dictate execution order

      \emph{within} largest set of shared loops
  \end{itemize}
\end{frame}
% -----------------------------------------------------------------------------
\begin{frame}{Loo.py vs reality}
  \begin{itemize}
    \item
    Two modes of operation:
    \begin{itemize}
      \item Standalone
      \item In-process
    \end{itemize}
    \item Flat data structure:
      \begin{itemize}
        \item Easy to manipulate
        \item Kernel fusion
      \end{itemize}
    \item Register-your-own:
      \begin{itemize}
        \item Functions
        \item Symbols
        \item Reductions
      \end{itemize}
    \item Literal code `escape hatch'
    \item Predicated execution
    \item Tree-of-domains for data-dependent control flow
  \end{itemize}
\end{frame}
% -----------------------------------------------------------------------------
\begin{frame}{Bonus Features}
  \begin{columns}
    \column{0.2\textwidth}
      \includegraphics[width=\textwidth]{glass-dollar.jpeg}
    \column{0.7\textwidth}
      Free extras:
      \begin{itemize}
        \item A-priori bounds checking
        \item Generate a sequential version of the code
        \item Automatic Benchmarking
        \item Free tuning advice
          \begin{itemize}
            \item Local memory layout
            \item Suboptimal use of hw parallelism
            \item Based on knowledge about target hardware
          \end{itemize}
        \item Automatic Testing
          \begin{itemize}
            \item \dots against sequential version
            \item \dots which is easier to verify
          \end{itemize}
      \end{itemize}
  \end{columns}
\end{frame}
\addimgcredit{Glass dollar: sxc.hu/flaivoloka}
% -----------------------------------------------------------------------------
\begin{frame}
  \bigncentered{DEMO TIME}
\end{frame}

% }}}
\end{document}

% vim: foldmethod=marker
